\lstset{language=file,frame=single}

El archivo \texttt{curso.txt} contiene los datos
de los estudiantes de un curso.
Cada línea corres\-ponde a un estudiante,
y contiene varios datos separados por dos puntos («\verb+:+»)
como en el si\-guiente ejemplo:
\lstinputlisting[linerange=-5]{notas-curso/curso.txt}

Los dos primeros datos son el nombre y el apellido.
A conti\-nua\-ción está la fecha de nacimiento
en formato \texttt{año/mes/día}.
Finalmente, están las cinco notas obtenidas por el estudiante.

\newpage
\begin{certamen}

    \item
      Escriba un programa llamado \texttt{clasificar.py} que,
      a partir de los datos de \texttt{curso.txt},
      genere dos nuevos archivos
      llamados \texttt{aprobados.txt} y \texttt{reprobados.txt},
      que tengan los datos de los estudiantes aprobados y reprobados,
      respectivamente,
      usando el formato \texttt{nombre:apellido:promedio}.

      Por ejemplo,
      las primeras líneas de \texttt{aprobados.txt} serían éstas:
      \lstinputlisting[linerange=-4]{notas-curso/aprobados.txt}
      y las de \texttt{reprobados.txt}:
      \lstinputlisting[linerange=-4]{notas-curso/reprobados.txt}

      \newpage

    \item
      Escriba un programa que genere un archivo llamado \texttt{promedios.txt}
      con un reporte como el siguiente:
      \lstinputlisting[linerange=-10]{notas-curso/promedios.txt}

      Note que los promedios están alineados a la derecha.
      Para lograr esto con el método \li!.format(...)!
      puede usar la siguiente plantilla:
      \begin{lstlisting}[language=py,frame=none]
plantilla = '{0:20} {1:>4} {2}\n'
      \end{lstlisting}

      \newpage

    \item
      Escriba el programa \texttt{reporte.py}
      que muestre la siguiente información por pantalla
      a partir de los datos del archivo:
      \lstinputlisting[language=testcase]{notas-curso/reporte.txt}

      \newpage

    \item
      Escriba el programa \texttt{pagina\_web.py}
      que genere un archivo llamado \texttt{curso.html}
      con un contenido como el siguiente:
      \lstinputlisting[basicstyle=\small\ttfamily]{notas-curso/ejemplo.html}
      (Por brevedad aquí sólo se muestra los primeros dos estu\-diantes;
      usted debe agregarlos todos).
      Los datos listados son:
      el nombre completo,
      las notas de los cinco certámenes
      y el promedio final (encerrado entre \verb+<b>+ y \verb+</b>+).

      El archivo generado es una página web.
      Ábralo con su nave\-gador favorito
      para verificar que fue creado correctamente.

\end{certamen}

