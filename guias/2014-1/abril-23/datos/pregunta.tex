Escriba un programa que muestre el promedio,
la desviación estándar y la mediana
de los datos ingresados por el usuario.
El programa debe verse así:

\begin{minipage}[t]{.60\textwidth}
  \lstinputlisting[language=testcase,frame=single]{datos/caso.txt}
\end{minipage}

La desviación estándar es la raíz cuadrada
del promedio de los cuadrados
de las diferencias entre cada dato y el promedio:
\[
  \sigma = \sqrt{
    \frac{1}{N} \sum_{i=0}^{N - 1}(x_i - \bar x)^2
  }
\]

La mediana es aquel dato para el que hay
la misma cantidad de otros datos mayores que él
y menores que él.
