El diccionario \verb!temp! asocia a cada ciudad
una tupla con sus temperaturas mínima y máxima del día:

\lstinputlisting[linerange=EJEMPLO-FIN\ EJEMPLO]{temperaturas/temperaturas.py}

\begin{enumerate}[leftmargin=0pt,label=\emph{\alph*})]
  \item
    Escriba la función \verb!rango_en(ciudad, temp)!
    que retorne cuál fue la dife\-rencia entre la máxima y la mínima
    en la ciudad indicada.

  \item
    Escriba la función \verb!mayor_temperatura(temp)!
    que haga lo obvio.

  \item
    Escriba la función \verb!ciudad_mas_fria(temp)!
    que retorne la ciudad en la que ocurrió la temperatura más fría.
\end{enumerate}

\lstinputlisting[linerange=CASOS-FIN\ CASOS]{temperaturas/temperaturas.py}
