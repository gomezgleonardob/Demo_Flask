Un analista financiero desea estudiar el comportamiento
del precio del dólar de los últimos días.
Para esto, le encargará a su equipo
el desarrollo de un programa que le facilite la tarea.

El programa debe comenzar preguntando
cuántos precios serán ingresados,
y a continuación debe preguntar esos precios.

\begin{enumerate}[leftmargin=0pt,start=0]

  \item
    Escriba el programa que pide al usuario ingresar los datos.

  \item
    Haga que su programa al final muestre
    cuál fue el promedio de los precios.
    ¿Cómo lo haría para que el precio sea mostrado
    siempre con una sola cifra decimal?

  \item
    Haga que su programa al final muestre
    cuántas veces el precio fue mayor que 490.

  \item
    Haga que su programa al final muestre
    la cantidad de alzas;
    es decir, cuántas veces el precio fue mayor
    al del día anterior.

    Si no hubo ninguna alza,
    el programa debe mostrar el mensaje
    \texttt{No hubo alzas}.

  \item
    Modifique su programa para que al final muestre
    cuál fue el precio más alto.

  \item
    Modifique su programa para que al final muestre
    cuál fue la mayor de las alzas.
    Si no hubo alzas, no debe mostrar nada.

\end{enumerate}

El siguiente es un ejemplo de cómo debería verse el programa terminado:

\begin{minipage}[t]{0.6\textwidth}
  \lstinputlisting[language=testcase,frame=single]{precio-dolar/caso.txt}
\end{minipage}

