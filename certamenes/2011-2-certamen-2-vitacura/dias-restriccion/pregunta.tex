La ciudad de Pitonia tiene una alta congestión de vehículos
circulando por sus calles.
Las autoridades han decidido aplicar restricción vehicular
para descongestionar la ciudad en base a las patentes de los vehículos.

La patente de un vehículo es un string de cuatro letras y dos dígitos,
y la restricción depende sólo del \emph{penúltimo} dígito.
Por ejemplo, para la patente \li!GEEA78!,
el dígito \li!7! es el utilizado para evaluar la restricción.

La restricción vehícular de los días hábiles de la semana
se encuentra ingresada en el diccionario \li!digitos!,
cuyas llaves son los días de la semanas,
y cuyos valores son tuplas de cuatro enteros que representan
los dígitos con restricción de ese día.
\begin{enumerate}
  \item Implemente la función \li!esta_con_restriccion(digitos, dia, patente)!,
    que indique si el vehículo está o no con restricción.
  \item Implemente la función \li!dias_con_restriccion(digitos, patente)!,
    que retorne la lista de los días en que el vehículo no puede circular.
  \item Implemente la función \li!dias_sin_restriccion(digitos, patente)!,
    que retorne el conjunto de los días en que el vehículo sí puede circular.
\end{enumerate}
\lstinputlisting[linerange=CASO-FIN\ CASO]{dias-restriccion/restriccion.py}

