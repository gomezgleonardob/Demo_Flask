El \emph{intercalao} es un juego muy popular
entre los niños de la aldea de Pythópolis.

El juego consiste en lanzar varias veces una moneda.
En cada lanzamiento,
el resultado puede ser cara (\li!C!) o sello (\li!S!).

Un jugador gana cuando durante cuatro lanzamientos consecutivos
aparecen caras y sellos intercalados
(es decir, ningún resultado aparece dos veces seguidas),
y pierde cuando un mismo resultado aparece
cuatro veces seguidas.

Escriba un programa que reciba como entrada
los resultados de todos los lanzamientos
hasta que termine el juego,
y le indique al usuario si ganó o perdió.

\begin{minipage}[t]{.26\textwidth}
  \lstinputlisting[language=testcase,frame=single,linerange=CASO\ 1-FIN\ CASO\ 1]{intercalao/casos-intercalao.txt}
\end{minipage}
\hspace{1em}
\begin{minipage}[t]{.26\textwidth}
  \lstinputlisting[language=testcase,frame=single,linerange=CASO\ 2-FIN\ CASO\ 2]{intercalao/casos-intercalao.txt}
\end{minipage}

