En un campo de batalla
hay objetivos que se desea destruir (\frownie).
Todos los objetivos están representados
como tuplas \verb+(x, y)+
indicando su posición en el campo.

\begin{minipage}{0.65\textwidth}
  A su vez,
  en el campo hay varias posiciones amigas (\smiley)
  que se desea proteger.
  Las posiciones amigas también están representadas
  como tuplas \verb+(x, y)+.
  \\[1ex]
  Durante la próxima misión,
  se lanzarán varias bombas sobre el campo de batalla.
  Cada bomba está representada
  como una tupla \verb+(x, y, r)+,
  donde \verb+x+ e \verb+y+ indican el punto de impacto
  y \verb+r+ es el radio de acción de la bomba.
  Al caer en el punto de impacto,
  una bomba destruye todo lo que está
  a una distancia menor o igual que \verb+r+.
\end{minipage}
\hfill
\begin{minipage}{.25\textwidth}
  \hfill
  \begin{tikzpicture}[scale=.4, auto]
    \tikzstyle{target}=[inner sep=2pt, circle, fill=gray!40]
    \tikzstyle{friend}=[inner sep=2pt, circle, fill=gray!90]
    \draw[dotted] (0, 0) grid (10, 10);
    \draw (0, 0) rectangle (10, 10);
    \node[friend] (f0) at (3.9, 2.1) {\smiley};
    \node[friend] (f1) at (8.7, 0.3) {\smiley};
    \node[friend] (f2) at (9.1, 8.2) {\smiley};
    \node[friend] (f3) at (0.9, 9.5) {\smiley};
    \node[target] (t0) at (5.5, 5.5) {\frownie};
    \node[target] (t1) at (4.5, 4.5) {\frownie};
    \node[target] (t2) at (1.4, 6.2) {\frownie};
    \node[target] (t3) at (7.7, 8.4) {\frownie};
    \node[target] (t4) at (9.1, 3.1) {\frownie};
    \draw (5.1, 3.7) circle (3cm);
    \draw[latex'-latex'] (5.1, 3.7) --
        node [midway, below] {\(r\)}
        +(30:3cm);
    \node[inner sep=1pt, fill, circle] at (5.1, 3.7) {};
  \end{tikzpicture}
\end{minipage}

\begin{enumerate}
  \item Escriba la función \verb+contar_impactos(bomba, objetivos, amigos)+.
    La función debe retornar dos valores:
    la cantidad de objetivos destruidos por la bomba
    y la cantidad de posiciones amigas destruidas por la bomba.
    Los parámetros \verb+objetivos+ y \verb+amigos+
    son listas de objetivos y posiciones amigas, respectivamente.

  \item
    Una bomba es útil si destruye por lo menos un objetivo
    y no destruye ninguna posición amiga.

    Escriba la función \verb+bombas_utiles(bombas, objetivos, amigos)+,
    que recibe como primer parámetro una lista de bombas,
    y retorna una nueva lista que contiene sólo las bombas que son útiles.
\end{enumerate}

