En finanzas,
el \emph{valor actual neto} es un indicador
de cuán rentable será un proyecto.

Se calcula sumando
los flujos de dinero de cada mes
divididos por \((1 + r)^n\),
donde \(n\) es el número del mes
y \(r\) es la tasa de descuento mensual,
y restando la inversión inicial.

Por ejemplo,
en un proyecto en que la inversión inicial es \(\$900\),
los flujos de dinero estimados para los primeros cuatro meses son
\(\$550\), \(\$230\), \(\$341\) y \(\$190\),
y la tasa de descuento mensual es de 4\%,
el valor actual neto es:
\[
  \text{VAN} = -900 +
               \frac{550}{(1 + 0.04)^1} +
               \frac{230}{(1 + 0.04)^2} +
               \frac{341}{(1 + 0.04)^3} +
               \frac{190}{(1 + 0.04)^4}.
\]

Si el VAN da negativo, entonces no es conveniente comenzar el proyecto.

\begin{minipage}[t]{.43\textwidth}
  Escriba un programa que pida al usuario ingresar la inversión inicial
  y el porcentaje de tasa de descuento.
  A continuación,
  debe preguntar el flujo de dinero estimado para cada mes
  y mostrar cuál es la parte entera del VAN hasta ese momento.

  El programa debe terminar apenas el VAN comience a dar positivo.

  \vspace{1ex}
  Suponga que todos los datos ingresados son válidos.
\end{minipage}
\hfill
\begin{minipage}[t]{.45\textwidth}
  \lstinputlisting[language=testcase,frame=single]{van/caso4.txt}
\end{minipage}

