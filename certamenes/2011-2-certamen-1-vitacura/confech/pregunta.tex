La CONFECH,
en su afán de agilizar el proceso de recuento de las votaciones,
le ha encargado el desarrollo de un programa de registro de votación por universidades.

\begin{minipage}[t]{.5\textwidth}
  Primero, el programa debe solicitar al usuario ingresar
  la cantidad de universidades que participan en el proceso.
  \vspace{1ex}

  Luego, para cada una de las universidades,
  el usuario debe ingresar
  el nombre de la universidad
  y los votos de sus alumnos, que pueden ser:
  \emph{aceptar} (\texttt{A}),
  \emph{rechazar} (\texttt{R}),
  \emph{nulo} (\texttt{N}) o
  \emph{blanco} (\texttt{B}).
  El término de la votación se indica ingresando una \texttt{X},
  tras lo cual se debe mostrar los totales de votos de la universidad,
  con el formato que se muestra en el ejemplo.
  \vspace{1ex}

  Finalmente,
  el programa debe mostrar el resultado de la votación,
  indicando la cantidad de universidades que aceptan, que rechazan
  y en las que hubo empate entre estas dos opciones.

\end{minipage}
\hfill
\begin{minipage}[t]{.4\textwidth}
  \lstinputlisting[language=testcase,frame=single,basicstyle=\tiny\ttfamily]{confech/caso-confech.txt}
\end{minipage}

