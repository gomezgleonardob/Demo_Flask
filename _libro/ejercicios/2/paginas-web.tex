\section{Páginas web}

Una página web es simplemente un archivo de texto con extensión
\lstinline!.html! cuyo contenido sigue un conjunto de reglas que
describen la estructura de un documento.

Como ejemplo, usted puede usar su navegador para ver el código de esta
misma página haciendo clic con el botón derecho y buscando

Para hacer una página web sencilla, usted puede crear un archivo como el
siguiente:

\begin{lstlisting}
<!doctype html>
<body>

  Mi primera pagina.

</body>
\end{lstlisting}

Haga la prueba: copie este texto en un editor de texto, guárdelo con el
nombre \lstinline!pagina.html!, y ábralo con su navegador web. Usted
debería poder ver el texto \lstinline!Mi primera pagina!.

\begin{enumerate}[1.]
\item
  Para crear una tabla en una página web, se hace de la siguiente
  manera:

\begin{lstlisting}
<table>
  <tr>
    <td> A </td>
    <td> B </td>
    <td> C </td>
  </tr>
  <tr>
    <td> D </td>
    <td> E </td>
    <td> F </td>
  </tr>
</table>
\end{lstlisting}

  Haga la prueba: en el archivo \lstinline!pagina,html!, reemplace la
  parte que dice \lstinline!Mi primera pagina! con el texto de arriba.
  Al abrir la página nuevamente en el navegador, usted debería ver una
  tabla de dos filas y tres columnas.

  Escriba un programa que genere una página web llamada
  \lstinline!multiplicacion.html! que tenga una tabla de multiplicar de
  20 × 20.
\end{enumerate}
