\section{Números romanos}

Los números romanos aún son utilizados para algunos propósitos.

Los símbolos básicos y sus equivalencias decimales son:

\ctable[pos = H, center, botcap]{ll}
{% notes
}
{% rows
\FL
\parbox[t]{0.06\columnwidth}{\raggedright
M
} & \parbox[t]{0.10\columnwidth}{\raggedright
1000
}
\\\noalign{\medskip}
\parbox[t]{0.06\columnwidth}{\raggedright
D
} & \parbox[t]{0.10\columnwidth}{\raggedright
\begin{quote}
500
\end{quote}
}
\\\noalign{\medskip}
\parbox[t]{0.06\columnwidth}{\raggedright
C
} & \parbox[t]{0.10\columnwidth}{\raggedright
\begin{quote}
100
\end{quote}
}
\\\noalign{\medskip}
\parbox[t]{0.06\columnwidth}{\raggedright
L
} & \parbox[t]{0.10\columnwidth}{\raggedright
\begin{quote}
50
\end{quote}
}
\\\noalign{\medskip}
\parbox[t]{0.06\columnwidth}{\raggedright
X
} & \parbox[t]{0.10\columnwidth}{\raggedright
\begin{quote}
10
\end{quote}
}
\\\noalign{\medskip}
\parbox[t]{0.06\columnwidth}{\raggedright
V
} & \parbox[t]{0.10\columnwidth}{\raggedright
\begin{quote}
5
\end{quote}
}
\\\noalign{\medskip}
\parbox[t]{0.06\columnwidth}{\raggedright
I
} & \parbox[t]{0.10\columnwidth}{\raggedright
\begin{quote}
1
\end{quote}
}
\LL
}

Los enteros romanos se escriben de acuerdo a las siguientes reglas:

\begin{itemize}
\item
  Si una letra está seguida inmediatamente por una de igual o menor
  valor, su valor se suma al total acumulado. Así, XX = 20, XV = 15 y VI
  = 6.
\item
  Si una letra está seguida inmediatamente por una de mayor valor, su
  valor se sustrae del total acumulado. Así, IV = 4, XL = 40 y CM = 900.
\end{itemize}

Escriba la función \lstinline!romano_a_arabigo! que reciba un string con
un número en notación romana, y entregue el entero equivalente:

\begin{lstlisting}
>>> romano_a_arabigo('MCMXIV')
1914
>>> romano_a_arabigo('XIV')
14
>>> romano_a_arabigo('X')
10
>>> romano_a_arabigo('IV')
4
>>> romano_a_arabigo('DLIV')
554
>>> romano_a_arabigo('CCCIII')
303
\end{lstlisting}

