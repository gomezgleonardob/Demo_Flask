\section{Notas de certamenes}

Desarrolle un programa para poder manipular las notas de tres certámenes
de los 10 alumnos de un curso:

\ctable[pos = H, center, botcap]{lll}
{% notes
}
{% rows
\FL
\parbox[b]{0.07\columnwidth}{\raggedright
C1
} & \parbox[b]{0.07\columnwidth}{\raggedright
C2
} & \parbox[b]{0.07\columnwidth}{\raggedright
C3
}
\ML
\parbox[t]{0.07\columnwidth}{\raggedright
58
} & \parbox[t]{0.07\columnwidth}{\raggedright
65
} & \parbox[t]{0.07\columnwidth}{\raggedright
53
}
\\\noalign{\medskip}
\parbox[t]{0.07\columnwidth}{\raggedright
44
} & \parbox[t]{0.07\columnwidth}{\raggedright
84
} & \parbox[t]{0.07\columnwidth}{\raggedright
60
}
\\\noalign{\medskip}
\parbox[t]{0.07\columnwidth}{\raggedright
\ldots{}
} & \parbox[t]{0.07\columnwidth}{\raggedright
\ldots{}
} & \parbox[t]{0.07\columnwidth}{\raggedright
\ldots{}
}
\\\noalign{\medskip}
\parbox[t]{0.07\columnwidth}{\raggedright
65
} & \parbox[t]{0.07\columnwidth}{\raggedright
60
} & \parbox[t]{0.07\columnwidth}{\raggedright
80
}
\LL
}

Para almacenar la información se puede utilizar una matriz de 10 filas y
3 columnas, donde en cada columna para una determinada fila, se
almacenará la nota que el alumno, representado por la fila, obtuvo en
uno de los tres certámenes (la fila \emph{i} representa al alumno
\emph{i} y la nota del certamen 1 está en la columna 1, la nota del
certamen 2 en la columna 2, etc).

Además, utilice un arreglo unidimensional de largo 10 para almacener los
nombres de los alumnos. Así en la posición \emph{i} del arreglo estará
el nombre del alumno \emph{i}. El programa debe:

\begin{itemize}
\item
  Ingresar el nombre y las notas para todos los alumnos del curso.
\item
  Calcular el promedio de cada alumno.
\item
  Calcular el promedio del curso en cada certamen.
\item
  Calcular el promedio general del curso.
\item
  Calcular la cantidad de alumnos aprobados (promedio \textgreater{}=
  55) y reprobados (promedio \textless{} 55
\item
  Ordenar alumnos según promedio
\item
  Mostrar el nombre, notas de certámenes y promedio final para cada
  alumno ordenado.
\end{itemize}

\begin{lstlisting}
Nombre 1: Manuel Pavez
C1: 55
C2: 60
C3: 75
Nombre 2: Guillermo Fuenzalida
C1: 60
C2: 60
C3: 20
...
Nombre 10: Cristian Perez
C1: 100
C2: 0
C3: 55
Promedio 1: 63.3
Promedio 2: 46.6
...
Promedio 10: 51.6
Promedio del curso C1: 56
Promedio del curso C2: 30
Promedio del curso C3: 60
Promedio Final Curso: 55
Aprobados: 4
Reprobados: 6
...
Guillermo Fuenzalida 46.6
...
Cristian Perez 51.6
...
Manuel Pavez 63.3
...
\end{lstlisting}

