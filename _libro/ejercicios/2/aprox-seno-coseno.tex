\section{Aproximación de seno y coseno}

La funciones seno y coseno puede ser representadas mediante sumas
infinitas:

\[\text{sen}(x) =
\frac{x^1}{1!} -
\frac{x^3}{3!} +
\frac{x^5}{5!} -
\frac{x^7}{7!} +
\cdots\]

\[\text{cos}(x) =
\frac{x^0}{0!} -
\frac{x^2}{2!} +
\frac{x^4}{4!} -
\frac{x^6}{6!} +
\cdots\]

(Éstas son las
\href{http://es.wikipedia.org/wiki/Serie\_de\_Taylor}{series de Taylor}
en torno a \(x=0\) de las funciones seno y coseno, que usted estudiará en
Matemáticas 2).

Los términos de ambas sumas son cada vez más pequeños, por lo que
tomando algunos de los primeros términos es posible obtener una buena
aproximación.

\begin{enumerate}
\item
  Escriba la función \lstinline!factorial_reciproco(n)!, que retorne el
  valor \(1/n!\)
\item
  Escriba la función \lstinline!signo(n)! que retorne \(1\) cuando
  \lstinline!n! es par y \(-1\) cuando \lstinline!n! es impar.
\item
  Escriba las funciones \lstinline!seno_aprox(x, m)! y
  \lstinline!coseno_aprox(x, m)! que aproximen respectivamente el seno y
  el coseno usando los \lstinline!m! primeros términos de las sumas
  correspondientes. Las funciones deben llamar a las funciones
  \lstinline!factorial_reciproco! y \lstinline!signo!.
\item
  Escriba la función \lstinline!error(f_exacta, f_aprox, m, x)! que
  entregue cuál es la diferencia entre el valor exacto de la función
  \lstinline!f_exacta! y su aproximación con \lstinline!m! términos
  usando la función \lstinline!f_aprox! en \(x = \text{\lstinline!x!}\).

  Por ejemplo, el error del seno en \(x=2\) al usar 20 términos se
  obtendría así:

\begin{lstlisting}
>>> from math import sin
>>> error(sin, seno_aprox, 20, 2)
\end{lstlisting}
\end{enumerate}
