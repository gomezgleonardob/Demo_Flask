\section{Conjunto potencia}

El \emph{conjunto potencia} de un conjunto \(S\) es el conjunto de todos
los subconjuntos de \(S\).
Por ejemplo, el conjunto potencia de \(\{1, 2, 3\}\) es:
\[
  \bigl\{
    \emptyset,
    \{1\},
    \{2\},
    \{3\},
    \{1, 2\},
    \{1, 3\},
    \{2, 3\},
    \{1, 2, 3\}
  \bigr\}
\]

En Python, un conjunto no puede contener a otros conjuntos, ya que no
puede tener elementos mutables, y los conjuntos lo son:
\begin{lstlisting}
>>> a = set()
>>> a.add({1, 2})        # :(
Traceback (most recent call last):
  File "<console>", line 1, in <module>
TypeError: unhashable type: 'set'
\end{lstlisting}

Lo que sí podemos crear es una lista de conjuntos:
\begin{lstlisting}
>>> l = list()
>>> l.append({1, 2})     # :)
>>> l
[set([1, 2])]
\end{lstlisting}

Escriba la función \lstinline!conjunto_potencia(s)! que reciba como
parámetro un conjunto cualquiera \lstinline!s! y retorne su «lista
potencia» (la lista de todos sus subconjuntos):
\begin{lstlisting}
>>> conjunto_potencia({6, 1, 4})
[set(), set([6]), set([1]), set([4]), set([6, 1]),
set([6, 4]), set([1, 4]), set([6, 1, 4])]
\end{lstlisting}

