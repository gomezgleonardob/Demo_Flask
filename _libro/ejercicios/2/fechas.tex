\section{Fechas}

Una fecha puede ser representada como una tupla
\lstinline!(anno, mes, dia)!.

\begin{enumerate}

  \item
    Escriba la función \lstinline!dia_siguiente(f)! que reciba como
    parámetro una fecha \lstinline!f! y entegue cuál es la fecha
    siguiente:
\begin{lstlisting}
>>> dia_siguiente((2011, 4, 11))
(2011, 4, 12)
>>> dia_siguiente((2011, 4, 30))
(2011, 5, 1)
>>> dia_siguiente((2011, 12, 31))
(2012, 1, 1)
\end{lstlisting}

    Como recomendación, dentro de su función cree una lista con la cantidad
    de días que tiene cada mes:
\begin{lstlisting}
dias_mes = [31, 28, 31, 30, 31, 30, 31, 31, 30, 31, 30, 31]
\end{lstlisting}

  \item
    Escriba la función \lstinline!dias_entre(f1, f2)! que retorne la
    cantidad de días que han transcurrido entre las fechas \lstinline!f1!
    y \lstinline!f2!:
\begin{lstlisting}
>>> hoy = (2011, 4, 11)
>>> navidad = (2011, 12, 25)
>>> dias_entre(hoy, navidad)
258
>>> dias_entre(hoy, hoy)
0
\end{lstlisting}

  \item
    Escriba un programa que le diga al usuario cuántos días de edad tiene:
\begin{lstlisting}[language=testcase]
Ingrese su fecha de nacimiento.
Dia: `14`
Mes: `5`
Anno: `1990`
Ingrese la fecha de hoy.
Dia: `20`
Mes: `4`
Anno: `2011`
Usted tiene 7646 dias de edad
\end{lstlisting}

\end{enumerate}
