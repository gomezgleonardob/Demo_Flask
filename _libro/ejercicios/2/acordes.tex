\section{Acordes}

En teoría musical, la escala cromática está formada por doce notas:
\begin{lstlisting}
notas = ['do', 'do#', 're', 're#', 'mi', 'fa',
         'fa#', 'sol', 'sol#', 'la', 'la#', 'si']
\end{lstlisting}
%
\begin{figure}
  \centering
  \documentclass{minimal}
\usepackage[pdftex,active,tightpage]{preview}
\usepackage[utf8]{inputenc}
\usepackage{mathpazo}
\usepackage{tikz}
\usetikzlibrary{calc,arrows,decorations}
\newcommand\sh{$^\sharp$}
\PreviewEnvironment{tikzpicture}

\begin{document}
\begin{tikzpicture}[scale=.8]

  \foreach \i in {0,...,6} {
    \draw
         (\i,       5)
      -- ({\i + 1}, 5)
      {[rounded corners]
        -- ({\i + 1}, 0)
        -- (\i,       0)
      }
      -- cycle;
  }
  \foreach\i in {1,2,4,5,6} {
    \fill[black]
         ({\i + .3}, 5.0)
      -- ({\i - .3}, 5.0)
      {[rounded corners]
        -- ({\i - .3}, 1.8)
        -- ({\i + .3}, 1.8)
      }
      -- cycle;
  }

  \node at (0.5, .5) {do};  \node at (1, 5.3) {do\sh};
  \node at (1.5, .5) {re};  \node at (2, 5.3) {re\sh};
  \node at (2.5, .5) {mi};
  \node at (3.5, .5) {fa};  \node at (4, 5.3) {fa\sh};
  \node at (4.5, .5) {sol}; \node at (5, 5.3) {sol\sh};
  \node at (5.5, .5) {la};  \node at (6, 5.3) {la\sh};
  \node at (6.5, .5) {si};

\end{tikzpicture}

\end{document}


  \caption{Correspondencia entre notas y teclas del piano.}
  \label{fig:piano}
\end{figure}
%
(El signo \(\sharp\) se lee «sostenido»).
Cada nota corresponde a una tecla del piano,
como se muestra en la figura~\ref{fig:piano}.
Los sostenidos son las teclas negras.
La escala es circular, y se extiende infinitamente en ambas direcciones.
Esto signfica que después de si viene nuevamente do.

Cada par de notas consecutivas está separada por un semitono. Por
ejemplo, entre re y sol\({}^\sharp\) hay 6 semitonos.

Un \emph{acorde} es una combinación
de notas que suenan bien al unísono.

Existen varios tipos de acordes, que difieren en la cantidad de
semitonos por las que sus notas están separadas.
Por ejemplo, los acordes mayores tienen tres notas separadas por 4 y 3
semitonos. Así es como el acorde de re mayor está formado por las notas:
\begin{quote}
re, fa\({}^\sharp\) y la,
\end{quote}
pues entre re y fa\({}^\sharp\) hay 4 semitonos,
y entre fa\({}^\sharp\) y la, 3.

Algunos tipos de acordes están presentados en el siguiente diccionario,
asociados a las separaciones entre notas consecutivas del acorde:

\begin{lstlisting}
acordes = {
    'mayor': (4, 3),
    'menor': (3, 4),
    'aumentado': (4, 4),
    'disminuido': (3, 3),
    'sus 2': (2, 5),
    'sus 4': (5, 2),
    '5': (7,),
    'menor 7': (3, 4, 3),
    'mayor 7': (4, 3, 4),
    '7': (4, 3, 3),
}
\end{lstlisting}

Escriba la función \lstinline!acorde(nota, tipo)! que entegue una lista
de las notas del acorde en el orden correcto:

\begin{lstlisting}
>>> acorde('la', 'mayor')
['la', 'do#', 'mi']
>>> acorde('sol#', 'menor')
['sol#', 'si', 'do#']
>>> acorde('si', '7')
['si', 're#', 'fa#', 'la']
>>> acorde('do#', '5')
['do#', 'sol#']
\end{lstlisting}

Si el tipo no es entregado, la función debe suponer que el acorde es
mayor:
\begin{lstlisting}
>>> acorde('si')
['si', 're#', 'fa#']
\end{lstlisting}

