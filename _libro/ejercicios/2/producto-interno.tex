\section{Producto interno}

El \emph{producto interno} de dos listas de números es la suma de los
productos de los términos correspondientes de ambas.
Por ejemplo, si:
\begin{lstlisting}
a = [5, 1, 6]
b = [1, -2, 8]
\end{lstlisting}
entonces el producto interno entre \lstinline!a! y \lstinline!b! es:
\begin{lstlisting}
(5 * 1) + (1 * -2) + (6 * 8)
\end{lstlisting}

\begin{enumerate}

\item
  Escriba la función \lstinline!producto_interno(a, b)! que entregue el
  producto interno de \lstinline!a! y \lstinline!b!:

\begin{lstlisting}
>>> a = [7, 1, 4, 9, 8]
>>> b = range(5)
>>> producto_interno(a, b)
68
\end{lstlisting}

\item
  Dos listas de números son \emph{ortogonales} si su producto interno
  es cero. Escriba la función \lstinline!son_ortogonales(a, b)! que
  indique si \lstinline!a! y \lstinline!b! son ortogonales:
\begin{lstlisting}
>>> son_ortogonales([2, 1], [-3, 6])
True
\end{lstlisting}
\end{enumerate}
