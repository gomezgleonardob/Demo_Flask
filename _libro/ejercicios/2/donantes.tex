\section{Donantes}

Una institución de beneficiencia tiene un registro de las personas que
han hecho donaciones en un archivo de registros llamado
\lstinline!donantes.txt!.

El archivo está ordenado por rut de menor a mayor. Para simplificar,
vamos a suponer que los ruts tienen cinco dígitos, y no incluyen el
dígito después de la raya.

Por ejemplo, el contenido del archivo puede ser el siguiente:

\begin{quote}
\ctable[pos = H, center, botcap]{lll}
{% notes
}
{% rows
\FL
Rut & Nombre & Monto
\ML
15274 & Fulana de Tal & 200
\\\noalign{\medskip}
15891 & Jean Dupont & 150
\\\noalign{\medskip}
16443 & Erika Mustermann & 400
\\\noalign{\medskip}
16504 & Perico Los Palotes & 80
\\\noalign{\medskip}
17004 & Jan Kowalski & 200
\LL
}
\end{quote}

Los problemas son los siguientes:

\begin{enumerate}[1.]
\item
  Escribir una función que cree el archivo con los datos de la tabla.
\item
  Escribir una función que muestre los datos del archivo.
\item
  Escribir una función que pida al usuario ingresar un rut, y muestre
  como salida el monto donado por esa persona.
\item
  Escribir una función que pida al usuario ingresar un rut, y elimine
  del archivo al donante con ese rut.
\item
  Escribir un programa que pida al usuario ingresar los datos de un
  donante, y los agregue al archivo.
\end{enumerate}
