\section{Checkouts de dardos}

Un tablero de dardos está dividido en 20 sectores, numerados del 1 al
20. Al tirar un dardo en uno de los sectores, el puntaje obtenido es el
número asociado al sector.

Cada sector tiene una zona de doble puntaje (anillo exterior) y una de
triple puntaje (anillo central). El centro del tablero (llamado
\emph{bull}) entrega 50 puntos, y el anillo alrededor de él, 25.

\includegraphics{../../diagramas/dardos.png}

En cada juego de una partida de dardos, ambos jugadores comienzan con
501 puntos. En cada turno, un jugador lanza tres dardos, y descuenta de
su total el puntaje obtenido.

El objetivo es llegar exactamente a 0 puntos antes que el contrario. La
única restricción es que el último dardo debe ser un doble (anillo
exterior) o el \emph{bull}.

Escriba un programa que reciba como entrada el puntaje que le queda a un
jugador, y entregue todas las combinaciones con las que puede lograr
exactamente ese puntaje lanzando tres dardos o menos, bajo la
restricción que el último debe ser un doble o un \emph{bull}.

Como referencia, para hacer 100 puntos hay 546 combinaciones (una de las
cuales es \lstinline!20 T20 D10!), y para hacer 132 puntos hay 60
combinaciones (una de las cuales es \lstinline!25 T19 BULL!).
