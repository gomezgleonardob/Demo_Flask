\section{ADN}

Debido a la gran cantidad de crímenes y casos sin resolver, la policía
local ha decidido implementar su propio sistema de reconocimiento de
sospechosos con la técnica basada en el uso del DNA.

Para esto la policía mantiene dos listas de información; la primera
contiene el nombre de las personas registradas en la región (nombre y
apellido), la otra, un cromosoma representativo del DNA de cada una de
esas personas.

Por simplicidad, un cromosoma se considera como una cadena de 0 (ceros)
y 1 (unos), de largo 20.

El método para determinar el sospechoso, es el siguiente:

\begin{itemize}
\item
  Se obtiene una muestra del cromosoma del autor del delito (20
  caracteres)
\item
  Se busca en la lista de cromosomas, aquel cromosoma que es \emph{más
  parecido} a la muestra. El más parecido se define como el cromosoma
  que tiene más genes (caracteres) iguales a la muestra.
\item
  Al terminar la búsqueda, se muestra el nombre de la persona cuyo
  cromosoma es sospechoso, con el porcentaje de parentesco.
\end{itemize}

La informacíon inicial del programa se debe ingresar directamente en el
código, es decir, nombres y cromosomas, en cambio la secuencia encontrda
en la escena del crimen, debe ser ingresada por el usuario.

Desarrolle un programa que lleve a cabo la búsqueda descrita a partir de
una muestra de entrada.

Recuerde que como se trata de dos listas, la información del
\emph{nombre} como la de los cromosomas, deben estar en la misma
posición en ambas listas.

Consideremos, personas como Pedro, Juan y Diego. Sus secuencias serán

\begin{itemize}
\item
  00000101010101010101
\item
  00101010101101110111
\item
  00100010010000001001
\end{itemize}

\begin{lstlisting}
Ingrese secuencia: 01010101000011001100
El culpable es Pedro con un parentezco de 60%   
\end{lstlisting}

