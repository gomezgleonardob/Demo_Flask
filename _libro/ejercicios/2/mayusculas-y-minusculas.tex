\section{Mayúsculas y minúsculas}

Cuando debemos trabajar con distintos textos, nos encontramos con
situaciones donde las mayúsculas y las minúsculas no son siempre bien
utilizadas.

Por lo general, cuando tenemos un párrafo de texto las reglas son que la
primera letra debe ser mayúscula y si es que hay algún punto seguido,
también.

Por otro lado, cuando nos referimos a nombres de personas o
instituciones, estas deben utilizar la primera letra de cada palabra, o
sin son siglas deben ir todas en mayúsculas.

Utilice \emph{conjuntos} para poder almacenar las siglas, nombres e
instituciones, y corrija el siguiente texto:

\begin{lstlisting}
La universidad se DEBE fundamentalmente a federico santa maría carrera,
Quien con una impORtante Donación la hizo pSIble y a Agustín Edwards Mc Clure,
Albacea de Santa María, Y, por ENde, ejecutor de su volunTAD Testamentaria
de dotar a Valparaíso de un centro de estudio compuesto de una Escuela de
ArteS y Oficios y un ColeGIO de ingenieros.
consiDERAndo las SUGerencias hechas por EDWARDS, en el sentido de asignarle
colaboradores para esa tarea, santa maría otorgó un testamento, cerrado en París,
con fecha 6 de eNero dE 1920.Reducidos a escRItura pública los estatutos de
la Fundación fedeRIco sANta maRía constituida por LOS albaceas, Fueron Aprobados
POR Decreto Supremo del 27 de abril de 1926.

eXtracto de LA reseña histórica, utfsm.
\end{lstlisting}

Considere las funciones \emph{upper()}, \emph{lower()} y
\emph{swapcase()}.
