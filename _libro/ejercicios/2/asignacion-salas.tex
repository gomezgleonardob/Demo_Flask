\section{Asignación de salas}

El edificio de una escuela secundaria tiene tres pisos, cada uno con
cinco salas de clases de tamaños diversos como se muestra en la
siguiente tabla:

\ctable[pos = H, center, botcap]{ll}
{% notes
}
{% rows
\FL
\parbox[b]{0.10\columnwidth}{\raggedright
} & \parbox[b]{0.35\columnwidth}{\raggedright
Número de sala
}
\ML
\parbox[t]{0.10\columnwidth}{\raggedright
Piso
} & \parbox[t]{0.35\columnwidth}{\raggedright
01 \textbar{} 02 \textbar{} 03 \textbar{} 04 \textbar{} 05
}
\\\noalign{\medskip}
\parbox[t]{0.10\columnwidth}{\raggedright
\begin{quote}
1
\end{quote}
} & \parbox[t]{0.35\columnwidth}{\raggedright
30 \textbar{} 30 \textbar{} 15 \textbar{} 30 \textbar{} 40
}
\\\noalign{\medskip}
\parbox[t]{0.10\columnwidth}{\raggedright
\begin{quote}
2
\end{quote}
} & \parbox[t]{0.35\columnwidth}{\raggedright
25 \textbar{} 30 \textbar{} 25 \textbar{} 10 \textbar{} 110
}
\\\noalign{\medskip}
\parbox[t]{0.10\columnwidth}{\raggedright
\begin{quote}
3
\end{quote}
} & \parbox[t]{0.35\columnwidth}{\raggedright
62 \textbar{} 30 \textbar{} 40 \textbar{} 40 \textbar{} 30
}
\LL
}

Cada semestre, la escuela imparte 15 cursos que deben distribuirse en
las salas.

Desarrolle un programa que intente hacer una asignación de salas
satisfactoria que acomode 15 clases, teniendo ya los datos de las
capacidaddes de salas y tambien el tamaño de cada grupo de los cursos.

Para los grupos que no puedan ser adecuadamente ubicados, el programa
mostrará el mensaje:

\begin{lstlisting}
No hay sala disponible
\end{lstlisting}

además, el programa indicará el número de asientos vacíos en cada sala y
en todo el edificio.

Para realiar la asignación utilice la estrategia del \emph{mejor
ajuste}: para una demanda dada se asigna la sala disponible cuyo exceso
de capacidad sea mínimo.
