\section{Cuadrado mágico}

\begin{figure}
  \centering
  \documentclass{minimal}
\usepackage[pdftex,active,tightpage]{preview}
\usepackage[utf8]{inputenc}
\usepackage[spanish]{babel}
\usepackage{mathpazo}
\usepackage{tikz}
\usetikzlibrary{calc,arrows,decorations,shapes}
\PreviewEnvironment{tikzpicture}

\begin{document}
\tikzstyle{n}=[xshift=.3cm, yshift=.3cm]
\tikzstyle{a}=[red]
\begin{tikzpicture}[scale=.6]
  %\foreach\i in {0,1,2,3}{
  %  \draw[a] (.1, {0 + .5}) -- ++(4.2, 0);
  %  \draw[a] ({0 + .5}, .1) -- ++(0, 4.2);
  %}

  \draw (0, 0) grid (4, 4);
  \node[n] at (0, 0) {16};
  \node[n] at (1, 0) {3};
  \node[n] at (2, 0) {2};
  \node[n] at (3, 0) {13};

  \node[n] at (0, 1) {5};
  \node[n] at (1, 1) {10};
  \node[n] at (2, 1) {11};
  \node[n] at (3, 1) {8};

  \node[n] at (0, 2) {9};
  \node[n] at (1, 2) {6};
  \node[n] at (2, 2) {7};
  \node[n] at (3, 2) {12};

  \node[n] at (0, 3) {4};
  \node[n] at (1, 3) {15};
  \node[n] at (2, 3) {14};
  \node[n] at (3, 3) {1};

  %\foreach\i in {0,1,2,3}{
  %  \node[n, a] at (4,  \i) { 34};
  %  \node[n, a] at (\i, -1) { 34};
  %}
  %\node[n, a] at (4, 4) { 34};
  %\node[n, a] at (4, -1) { 34};


\end{tikzpicture}
\end{document}

  \caption{Cuadrado mágico de \(4\times 4\).}
  \label{fig:magico}
\end{figure}

Un \emph{cuadrado mágico} es una disposición de números naturales en una
tabla cuadrada, de modo que las sumas de cada columna, de cada fila y de
cada diagonal son iguales.
Por ejemplo, todas las filas, columnas y diagonales
del cuadrado mágico de la figura~\ref{fig:magico} suman 34.


Los cuadrados mágicos más populares son aquellos que tienen los números
consecutivos desde el 1 hasta \(n^2\), donde \(n\) es el número de filas
y de columnas del cuadrado.

\begin{enumerate}
\item
  Escriba una función que reciba un arreglo cuadrado de enteros de
  \(n \times n\), e indique si está conformado por los números consecutivos
  desde 1 hasta \(n^2\):

\begin{lstlisting}
>>> from numpy import array
>>> consecutivos(array([[3, 1, 5],
...                     [4, 7, 2],
...                     [9, 8, 6]]))
True
>>> consecutivos(array([[3, 1, 4],
...                     [4, 0, 2],
...                     [9, 9, 6]]))
False
\end{lstlisting}
\item
  Escriba una función que reciba un arreglo e indique si se trata o no
  de un cuadrado mágico:

\begin{lstlisting}
>>> es_magico(array([[3, 1, 5],
...                  [4, 7, 2],
...                  [9, 8, 6]]))
False
>>> es_magico(array([[2, 7, 6],
...                  [9, 5, 1],
...                  [4, 3, 8]]))
True
\end{lstlisting}
\end{enumerate}
