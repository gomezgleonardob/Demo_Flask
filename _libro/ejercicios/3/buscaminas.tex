\section{Buscaminas}

El juego del buscaminas se basa en una grilla rectangular que representa
un campo minado. Algunas de las casillas de la grilla tienen una mina, y
otras no. El juego consiste en descubrir todas las casillas que no
tienen minas.

En un programa, podemos representar un campo de buscaminas como un
arreglo en el que las casillas minadas están marcadas con el valor \(-1\), y
las demás casillas con el valor 0:

\begin{lstlisting}
>>> from numpy import *
>>> campo = array([[ 0,  0, -1,  0,  0,  0,  0,  0],
                   [-1,  0,  0,  0, -1,  0,  0,  0],
                   [ 0,  0,  0,  0, -1,  0,  0, -1],
                   [ 0,  0, -1,  0,  0,  0,  0,  0],
                   [ 0,  0,  0,  0,  0,  0, -1,  0],
                   [ 0, -1,  0,  0, -1,  0,  0,  0],
                   [ 0,  0, -1,  0,  0,  0,  0,  0],
                   [ 0,  0,  0,  0,  0,  0,  0,  0]])
\end{lstlisting}

\begin{enumerate}
\item
  Escriba la función \lstinline!crear_campo(forma, n)!,
  donde \lstinline!forma! es una tupla \lstinline!(filas, columnas)!, que
  retorne un nuevo campo aleatorio con la forma indicada que tenga
  \lstinline!n! minas.

  Hágalo en los siguientes pasos:

  \begin{enumerate}%[a.]
  \item
    Construya un vector de tamaño \lstinline!filas * columnas! que tenga
    \lstinline!n! veces el valor \(-1\), y a continuación sólo ceros.
  \item
    Importe la función \lstinline!shuffle! desde el módulo
    \lstinline!numpy.random!. Esta función desordena (o «baraja») los
    elementos de un arreglo.
  \item
    Desordene los elementos del vector que creó.
  \item
    Cambie la forma del vector.
  \end{enumerate}

\begin{lstlisting}
>>> crear_campo((4, 4), 5)
array([[-1,  0,  0,  0],
       [ 0,  0,  0,  0],
       [ 0, -1, -1,  0],
       [ 0, -1, -1,  0]])
>>> crear_campo((4, 4), 5)
array([[ 0,  0, -1,  0],
       [ 0,  0,  0, -1],
       [-1,  0,  0,  0],
       [ 0,  0, -1, -1]])
>>> crear_campo((4, 4), 5)
array([[ 0,  0,  0, -1],
       [ 0,  0, -1, -1],
       [-1,  0,  0,  0],
       [ 0,  0, -1,  0]])
\end{lstlisting}
\item
  Al descubrir una casilla no minada, en ella aparece un número, que
  indica la cantidad de minas que hay en sus ocho casillas vecinas.

  Escriba la función \lstinline!descubrir(campo)! que modifique el campo
  poniendo en cada casilla la cantidad de minas vecinas:

\begin{lstlisting}
>>> c = crear_campo((4, 4), 5)
>>> c
array([[ 0,  0, -1, -1],
       [ 0,  0, -1,  0],
       [ 0,  0,  0, -1],
       [ 0,  0,  0, -1]])
>>> descubrir(c)
>>> c
array([[ 0,  2, -1, -1],
       [ 0,  2, -1,  4],
       [ 0,  1,  3, -1],
       [ 0,  0,  2, -1]])
\end{lstlisting}
\end{enumerate}
