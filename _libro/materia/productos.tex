\chapter{Productos entre arreglos}

Recordemos que \textbf{vector} es sinónimo de arreglo de una dimensión,
y \textbf{matriz} es sinónimo de arreglo de dos dimensiones.

\section{Producto interno (vector-vector)}

El \textbf{producto interno} entre dos vectores es la suma de los
productos entre elementos correspondientes:

\includegraphics{../diagramas/producto-interno.png}

El producto interno entre dos vectores se obtiene usando la función
\lstinline!dot! provista por NumPy:

\begin{lstlisting}
>>> a = array([-2.8 , -0.88,  2.76,  1.3 ,  4.43])
>>> b = array([ 0.25, -1.58,  1.32, -0.34, -4.22])
>>> dot(a, b)
-14.803
\end{lstlisting}

El producto interno es una operación muy común. Por ejemplo, suele
usarse para calcular totales:

\begin{lstlisting}
>>> precios = array([200, 100, 500, 400, 400, 150])
>>> cantidades = array([1, 0, 0, 2, 1, 0])
>>> total_a_pagar = dot(precios, cantidades)
>>> total_a_pagar
1400
\end{lstlisting}

También se usa para calcular promedios ponderados:

\begin{lstlisting}
>>> notas = array([45, 98, 32])
>>> ponderaciones = array([30, 30, 40]) / 100.
>>> nota_final = dot(notas, ponderaciones)
>>> nota_final
55.7
\end{lstlisting}

\section{Producto matriz-vector}

El \textbf{producto matriz-vector} es el vector de los productos
internos El producto matriz-vector puede ser visto simplemente como
varios productos internos calculados de una sola vez.

Esta operación también es obtenida usando la función \lstinline!dot!
entre las filas de la matriz y el vector:

\includegraphics{../diagramas/matriz-vector.png}

El producto matriz-vector puede ser visto simplemente como varios
productos internos calculados de una sola vez.

Esta operación también es obtenida usando la función \lstinline!dot!:

\begin{lstlisting}
>>> a = array([[-0.6,  4.8, -1.2],
               [-2. , -3.6, -2.1],
               [ 1.7,  4.9,  0. ]])
>>> x = array([-0.6, -2. ,  1.7])
>>> dot(a, x)
array([-11.28,   4.83, -10.82])
\end{lstlisting}

\section{Producto matriz-matriz}

El \textbf{producto matriz-matriz} es la matriz de los productos
internos entre las filas de la primera matriz y las columnas de la
segunda:

\includegraphics{../diagramas/matriz-matriz.png}

Esta operación también es obtenida usando la función \lstinline!dot!:

\begin{lstlisting}
>>> a = array([[ 2,  8],
               [-3,  7],
               [-8, -5]])
>>> b array([[-3, -5, -6, -3],
             [-9, -2,  3, -3]])
>>> dot(a, b)
array([[-78, -26,  12, -30],
       [-54,   1,  39, -12],
       [ 69,  50,  33,  39]])
\end{lstlisting}

La multiplicación de matrices puede ser vista como varios productos
matriz-vector (usando como vectores todas las filas de la segunda
matriz), calculados de una sola vez.

En resumen, al usar la función \lstinline!dot!, la estructura del
resultado depende de cuáles son los parámetros pasados:

\begin{lstlisting}
dot(vector, vector) → número
dot(matriz, vector) → vector
dot(matriz, matriz) → matriz
\end{lstlisting}

