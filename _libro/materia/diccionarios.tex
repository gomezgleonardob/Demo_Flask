\chapter{Diccionarios}

Un \textbf{diccionario} es un tipo de datos que sirve para asociar pares
de objetos.

Un diccionario puede ser visto como una colección de \textbf{llaves},
cada una de las cuales tiene asociada un \textbf{valor}. Las llaves no
están ordenadas y no hay llaves repetidas. La única manera de acceder a
un valor es a través de su llave.

\section{Cómo crear diccionarios}

Los diccionarios literales se crean usando paréntesis de llave
(\lstinline!{! y \lstinline!}!).
La llave y el valor van separados por dos puntos:
\begin{lstlisting}
>>> telefonos = {'Pepito': 5552437, 'Jaimito': 5551428,
...              'Yayita': 5550012}
\end{lstlisting}

En este ejemplo, las llaves son \lstinline!'Pepito'!,
\lstinline!'Jaimito'! y \lstinline!'Yayita'!, y los valores asociados a
ellas son, respectivamente, \lstinline!5552437!, \lstinline!5551428! y
\lstinline!5550012!.

Un diccionario vacío puede ser creado usando \lstinline!{}! o con la
función \lstinline!dict()!:

\begin{lstlisting}
>>> d = {}
>>> d = dict()
\end{lstlisting}

\section{Cómo usar un diccionario}

El valor asociado a la llave \lstinline!k! en el diccionario
\lstinline!d! se puede obtener mediante \lstinline!d[k]!:

\begin{lstlisting}
>>> telefonos['Pepito']
5552437
>>> telefonos['Jaimito']
5551428
\end{lstlisting}

A diferencia de los índices de las listas, las llaves de los
diccionarios no necesitan ser números enteros.

Si la llave no está presente en el diccionario, ocurre un \textbf{error
de llave} (\lstinline!KeyError!):

\begin{lstlisting}
>>> telefonos['Fulanito']
Traceback (most recent call last):
  File "<stdin>", line 1, in <module>
KeyError: 'Fulanito'
\end{lstlisting}

Se puede agregar una llave nueva simplemente asignándole un valor:

\begin{lstlisting}
>>> telefonos['Susanita'] = 4448139
>>> telefonos
{'Pepito': 5552437, 'Susanita': 4448139, 'Jaimito': 5551428,
'Yayita': 5550012}
\end{lstlisting}

Note que el orden en que quedan las llaves en el diccionario no es
necesa\-ria\-mente el mismo orden en que fueron agregadas.

Si se asigna un valor a una llave que ya estaba en el diccionario, el
valor anterior se sobreescribe. Recuerde que un diccionario no puede
tener llaves repetidas:

\begin{lstlisting}
>>> telefonos
{'Pepito': 5552437, 'Susanita': 4448139, 'Jaimito': 5551428,
'Yayita': 5550012}
>>> telefonos['Jaimito'] = 4448139
>>> telefonos
{'Pepito': 5552437, 'Susanita': 4448139, 'Jaimito': 4448139,
'Yayita': 5550012}
\end{lstlisting}

Los valores sí pueden estar repetidos. En el ejemplo anterior, Jaimito y
Susanita tienen el mismo número.

Para borrar una llave, se puede usar la sentencia \lstinline!del!:

\begin{lstlisting}
>>> del telefonos['Yayita']
>>> telefonos
{'Pepito': 5552437, 'Susanita': 4448139, 'Jaimito': 4448139}
\end{lstlisting}

Los diccionarios son iterables. Al iterar sobre un diccionario en un
ciclo \lstinline!for!, se obtiene las llaves:

\begin{lstlisting}
>>> for k in telefonos:
...     print k
...
Pepito
Susanita
Jaimito
\end{lstlisting}

Para iterar sobre las llaves, se usa \lstinline!d.values()!:

\begin{lstlisting}
>>> for v in telefonos.values():
...     print v
...
5552437
4448139
4448139
\end{lstlisting}

Para iterar sobre las llaves y los valores simultáneamente, se usa el
método \lstinline!d.items()!:

\begin{lstlisting}
>>> for k, v in telefonos.items():
...     print 'El telefono de', k, 'es', v
...
El telefono de Pepito es 5552437
El telefono de Susanita es 4448139
El telefono de Jaimito es 4448139
\end{lstlisting}

También es posible crear listas de llaves o valores:

\begin{lstlisting}
>>> list(telefonos)
['Pepito', 'Susanita', 'Jaimito']
>>> list(telefonos.values())
[5552437, 4448139, 4448139]
\end{lstlisting}

\lstinline!len(d)! entrega cuántos pares llave-valor hay en el
diccionario:

\begin{lstlisting}
>>> numeros = {15: 'quince', 24: 'veinticuatro'}
>>> len(numeros)
2
>>> len({})
0
\end{lstlisting}

\lstinline!k in d! permite saber si la llave \lstinline!k! está en el
diccionario \lstinline!d!:

\begin{lstlisting}
>>> patas = {'gato': 4, 'humano': 2, 'pulpo': 8,
...          'perro': 4, 'ciempies': 100}
>>> 'perro' in patas
True
>>> 'gusano' in patas
False
\end{lstlisting}

Para saber si una llave \emph{no} está en el diccionario, se usa el
operador \lstinline!not in!:

\begin{lstlisting}
>>> 'gusano' not in patas
True
\end{lstlisting}

\section{Restricciones sobre las llaves}

No se puede usar cualquier objeto como llave de un diccionario. Las
llaves deben ser de un tipo de datos inmutable. Por ejemplo, no se puede
usar listas:

\begin{lstlisting}
>>> d = {[1, 2, 3]: 'hola'}
Traceback (most recent call last):
  File "<console>", line 1, in <module>
TypeError: unhashable type: 'list'
\end{lstlisting}

Típicamente, las llaves de los diccionarios suelen ser números, tuplas y strings.
