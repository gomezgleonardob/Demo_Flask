\chapter{Algoritmos}

Un \textbf{algoritmo} es una secuencia de pasos para resolver un
problema.
Los pasos deben estar muy bien definidos, y tienen que describir sin
ambigüedades cómo llegar desde el inicio hasta el final.

\section{Componentes de un algoritmo}

Conceptualmente, un algoritmo tiene tres componentes:

\begin{enumerate}
\item
  la \textbf{entrada}: son los datos sobre los que el algoritmo opera;
\item
  el \textbf{proceso}: son los pasos que hay que seguir, utilizando la
  entrada;
\item
  la \textbf{salida}: es el resultado que entrega el algoritmo.
\end{enumerate}

El proceso es una secuencia de \textbf{sentencias}, que debe ser
realizada en orden. El proceso también puede tener \textbf{ciclos}
(grupos de sentencias que son ejecutadas varias veces) y
\textbf{condicionales} (grupos de sentencias que sólo son ejecutadas
bajo ciertas condiciones).

\section{Cómo describir un algoritmo}

Consideremos un ejemplo sencillo: un algoritmo para resolver ecuaciones
cuadráticas.

Una
\href{http://es.wikipedia.org/wiki/Ecuaci\%C3\%B3n\_de\_segundo\_grado}{ecuación
cuadrática} es una ecuación de la forma \(ax^2 + bx + c = 0\), donde
\(a\), \(b\) y \(c\) son datos dados, con \(a\ne 0\), y \(x\) es la incógnita cuyo
valor que se desea determinar.

Por ejemplo, \(2x^2 - 5x + 2 = 0\) es una ecuación cuadrática con \(a =
2\), \(b = -5\) y \(c = 2\). Sus soluciones son \(x_1 = 1/2\) y \(x_2 = 2\),
como se puede comprobar fácilmente al reemplazar estos valores en la
ecuación. El problema es cómo obtener estos valores en primer lugar.

El problema computacional de resolver una ecuación cuadrática puede ser
planteado así:

\begin{quote}
Dados \(a\), \(b\) y \(c\), encontrar los valores reales de \(x\) que satisfacen
\(ax^2 + bx + c = 0\).
\end{quote}

La entrada del algoritmo, pues, son los valores \(a\), \(b\) y \(c\), y la
salida son las raíces reales \(x\) (que pueden ser cero, una o dos) de la
ecuación. En un programa computacional, los valores de \(a\), \(b\) y \(c\)
deberían ser ingresados usando el teclado, y las soluciones \(x\) deberían
ser mostradas a continuación en la pantalla.

Al estudiar álgebra aprendemos un algoritmo para resolver este problema.
Es lo suficientemente detallado para que pueda usarlo cualquier persona,
(incluso sin saber qué es una ecuación cuadrática) o para que lo pueda
hacer un computador. A continuación veremos algunas maneras de describir
el procedimiento.

\subsection{Lenguaje natural}

Durante el proceso mental de diseñar un algoritmo, es común pensar y
describir los pasos en la misma manera en que hablamos a diario. Por
ejemplo:

\begin{quote}
Teniendo los valores de \(a\), \(b\) y \(c\), calcular el discriminante
\(\Delta = b^2 - 4ac\). Si es discriminante es negativo, entonces la ecuación no
tiene soluciones reales. Si es discriminante es igual a cero, entonces
la ecuación tiene una única solución real, que es \(x = -b/2a\). Si el
discriminante es positivo, entonces la ecuación tiene dos soluciones
reales, que son \(x_1 = (-b - \sqrt{\Delta})/2a\) y \(x_2 = (-b + \sqrt{\Delta})/2a\).
\end{quote}

Esta manera de expresar un algoritmo no es ideal, ya que el lenguaje
natural es:

\begin{itemize}
\item
  impreciso: puede tener ambigüedades;
\item
  no universal: personas distintas describirán el proceso de maneras
  distintas; y
\item
  no estructurado: la descripción no está expresada en función de
  componentes simples.
\end{itemize}

Aún así, es posible identificar los pasos del algoritmo. Por ejemplo,
hay que evaluar la expresión \(b^2 - 4ac\), y ponerle el nombre \(\Delta\) a
su resultado. Esto se llama \textbf{asignación}, y es un tipo de
instrucción que aparece en casi todos los algoritmos. Después de eso, el
algoritmo puede usar el nombre \(\Delta\) para referirse al valor calculado.

\subsection{Diagrama de flujo}

Un \textbf{diagrama de flujo} es una representación gráfica de un
algoritmo. Los pasos son representados por varios tipos de bloques, y el
flujo de ejecución es indicado por flechas que conectan los bloques,
tal como se muestra en la figura~\ref{fig:diagrama-flujo-cuadratica}.

\begin{figure}
  \centering
  \documentclass{minimal}
\usepackage[pdftex,active,tightpage]{preview}
\usepackage[utf8]{inputenc}
\usepackage[spanish]{babel}
\usepackage{mathpazo}
\usepackage{tikz}
\usetikzlibrary{calc,shapes,arrows}
\PreviewEnvironment{tikzpicture}

\newcommand{\str}[1]{\emph{``#1''}}

\begin{document}
\tikzstyle{decision} = [
  diamond,
  very thick,
  draw=red!50!black!50,
  %fill=red!20, 
  aspect=2,
  %text badly centered,
  top color=white,
  bottom color=red!50!black!20,
]
\tikzstyle{stmt} = [
  rectangle,
  very thick,
  draw=blue!50!black!50,
  %fill=blue!20, 
  text centered,
  minimum height=5ex,
  minimum width=5em,
  top color=white,
  bottom color=blue!50!black!20,
]
\tikzstyle{node} = [
  circle,
  very thick,
  draw=orange!50!black!50,
  fill=orange!20,
  minimum size=6ex,
]
\tikzstyle{io} = [
  very thick,
  draw=green!50!black!50,
  trapezium,
  trapezium left angle=80,
  trapezium right angle=-80,
  %fill=green!20!black!10,
  %rounded corners,
  %minimum height=5ex,
  top color=white,
  bottom color=green!50!black!20,
  text centered,
  minimum height=5ex,
  minimum width=5em,
]
\tikzstyle{conn} = [very thick, draw=black!50, -latex']



  \begin{tikzpicture}[node distance=9ex, auto]
    % Place nodes
    \node [node] (start) {inicio};
    \node [io, below of=start] (read) { Leer $a$, $b$ y $c$ };
    \node [stmt, below of=read] (disc) {$\Delta = b^2 - 4ac$};
    \node [decision, below of=disc] (neg)  {¿$\Delta < 0$?};
    \node [decision, below of=neg]  (zero) {¿$\Delta = 0$?};
    \node [stmt, below of=zero, text width=10em] (pos-sol) {
        $x_1 = (-b + \sqrt{\Delta})/2a$
        $x_2 = (-b - \sqrt{\Delta})/2a$
    };
    \node [stmt, left of=pos-sol, node distance=12em] (zero-sol) {
        $x_1 = -b/2a$
    };
    \node [io, below of=pos-sol,
               node distance=20ex,
               text width=18em] (pos-out) {
        Escribir \str{La primera solución es}, $x_1$ \\ %
        Escribir \str{La segunda solución es}, $x_2$
    };
    \node [io, below of=zero-sol, text width=10em] (zero-out) {
        Escribir \str{La única solución es}, $x_1$%
    };
    \node [io, right of=zero-out, text width=10em, node distance=23em] (neg-out) {
        Escribir \str{No hay soluciones}
    };
    \node [node, below of=pos-out] (end) {fin};

    \path [conn] (start) -- (read);
    \path [conn] (read)  -- (disc);
    \path [conn] (disc)  -- (neg);
    \path [conn] (neg.east)  -| node [very near start] {sí} (neg-out);
    \path [conn] (neg)       -- node [near start] {no} (zero);
    \path [conn] (zero.west) -| node [very near start] {sí} (zero-sol);
    \path [conn] (zero)      -- node [near start] {no} (pos-sol);
    \path [conn] (zero-sol) -- (zero-out);
    \path [conn] (pos-sol)  -- (pos-out);

    \node [inner sep=1pt] (m) at ($ (pos-out.south)!.5!(end.north) $) {};
    \path [conn] (pos-out.south)  -- (end);
    \path [conn] (neg-out.south)  |- (m);
    \path [conn] (zero-out.south) |- (m);
  \end{tikzpicture}

\end{document}

  \caption{Diagrama de flujo del algoritmo para resolver
    la ecuación cuadrática \(ax^2 + bx + c = 0\).}
  \label{fig:diagrama-flujo-cuadratica}
\end{figure}

El inicio y el final del algoritmo son representados con bloques
circulares. El algoritmo siempre debe ser capaz llegar desde uno hasta
el otro, sin importar por qué camino lo hace. Un algoritmo no puede
«quedarse pegado» en la mitad.

La entrada y la salida de datos son representadas con romboides,

Los diamantes representan condiciones en las que el algoritmo sigue uno
de dos caminos que están etiquetados con \emph{sí} o \emph{no},
dependiendo de si la condición es verdadera o falsa.

También puede haber ciclos, representados por flechas que regresan a
bloques anteriores. En este ejemplo, no hay ciclos.

Otras sentencias van dentro de rectángulos.
En este ejemplo, las sentencias son asignaciones,
representadas en la forma \lstinline!nombre = valor!.

Los diagramas de flujo no son usados en la práctica para programar, pero
son útiles para ilustrar cómo funcionan algoritmos sencillos.

\subsection{Pseudocódigo}

El \textbf{pseudocódigo} es una descripción estructurada de un algoritmo
basada en ciertas convenciones notacionales. Si bien es muy parecido al
código que finalmente se escribirá en el computador, el pseudocódigo
está pensado para ser leído por humanos.

\begin{figure}
  \centering
  \begin{verse}
    \textbf{leer} \(a\)\\
    \textbf{leer} \(b\)\\
    \textbf{leer} \(c\)

    \(\text{discriminante} = b^2 - 4ac\)

    \textbf{si} \(\text{discriminante} < 0\):\\
    \qquad\textbf{escribir} ``La ecuación no tiene soluciones reales''

    \textbf{o si no, si} \(\text{discriminante} = 0\):\\
    \qquad\(x = -b / 2a\)\\
    \qquad\textbf{escribir} ``La solución única es'', \(x\)

    \textbf{o si no}:\\
    \qquad\(x_1 = (-b - \sqrt{\text{discriminante}}) / 2a\)\\
    \qquad\(x_2 = (-b + \sqrt{\text{discriminante}}) / 2a\)\\
    \qquad\textbf{escribir} ``La primera solución real es:'', \(x_1\)\\
    \qquad\textbf{escribir} ``La segunda solución real es:'', \(x_2\)
  \end{verse}
  \caption{Pseudocódigo del algoritmo para resolver
    la ecuación cuadrática \(ax^2 + bx + c = 0\).}
  \label{fig:pseudocodigo-cuadratica}
\end{figure}

Una manera de escribir el algoritmo para la ecuación cuadrática en
pseudocódigo es la que se muestra en la figura~\ref{fig:pseudocodigo-cuadratica}.

Las líneas que comienzan con \textbf{leer} y \textbf{escribir}
denotan, respectivamente, la entrada y la salida del programa. Los
diferentes casos son representados usando sentencias \textbf{si} y
\textbf{o si no}. Las asignaciones siguen la misma notación que en el
caso de los diagramas de flujo.

La notación de pseudocódigo es bien liberal. Uno puede mezclar notación
de matemáticas y frases en español, siempre que quede absolutamente
claro para el lector qué representa cada una de las líneas del
algoritmo.

\subsection{Código}

El producto final de la programación siempre debe ser código que pueda
ser ejecutado en el computador. Esto requiere describir los algoritmos
en un \textbf{lenguaje de programación}. Los lenguajes de programación
definen un conjunto limitado de conceptos básicos, en función de los
cuales uno puede expresar cualquier algoritmo.

En esta asignatura, usaremos el lenguaje de programación
\href{http://python.org/}{Python} para escribir nuestros programas.

El código en Python para resolver la ecuación cuadrática es el
siguiente:
\lstinputlisting[language=py]{../_static/programas/cuadratica.py}

A partir de ahora, usted aprenderá a entender, escribir y ejecutar
códigos como éste.
