\chapter{El peor ajedrez del mundo}

El programa que analizaremos ahora es un sencillo juego de ajedrez. Este
código es más largo que los que hemos visto hasta ahora, así que cópielo
y péguelo en vez de tipearlo.

Cada pieza del ajedrez la representaremos con una letra:

\begin{itemize}
\item
  P es el peón,
\item
  T es la torre,
\item
  C es el caballo,
\item
  A es el alfil,
\item
  D es la dama, y
\item
  R es el rey.
\end{itemize}

Las piezas blancas serán letras mayúsculas, y las negras, minúsculas.

En cada turno, el programa mostrará la disposición del tablero, y pedirá
a uno de los jugadores que ingrese su jugada:

\begin{verbatim}
       0 1 2 3 4 5 6 7
      +---------------+
    a |t c . d r a . t|
    b |p p p . p p p p|
    c |. . . . . c . .|
    d |. . . p . . . .|
    e |. . A . P . a .|
    f |. . . . . D . .|
    g |P P P P . P P P|
    h |T C A . R . C T|
      +---------------+
    Juega blanco: 

\end{verbatim}

La jugada se ingresa indicando la casilla donde está la pieza que se
moverá, y la casilla a la que se moverá. Cada casilla se ingresa como
sus coordenadas (una letra y un número), y ambas casillas van separadas
por un espacio. Por ejemplo:

Nuestro juego de ajedrez es realmente malo. No hace cumplir las reglas,
por lo que se puede mover las piezas como a uno se le dé la gana.
¡Incluso el jugador blanco puede mover las piezas negras!

Si uno ingresa jugadas que no tengan sentido, el programa puede fallar
de maneras inesperadas. ¡Inténtelo!

\section{Tipos enumerados}

Un \textbf{tipo enumerado} es un tipo de datos que puede respresentar
sólo una lista de valores discretos indicados por el programador.

Para crear un tipo enumerado, en C se usa la sentencia \lstinline!enum!,
en la que se enumera cuáles son los valores posibles. Por ejemplo:

\begin{lstlisting}
enum sexo {MASCULINO, FEMENINO};
enum semaforo {ROJO, AMARILLO, VERDE};
enum palo {CORAZON, PICA, TREBOL, DIAMANTE};
\end{lstlisting}

Al principio de nuestro programa, creamos un tipo enumerado llamado
\lstinline!enum color!, que podemos usar cuando necesitemos guardar
algún color:

\begin{lstlisting}
enum color {BLANCO, NEGRO, VACIO};
\end{lstlisting}

El valor \lstinline!VACIO! nos permite usar variables
\lstinline!enum color!, por ejemplo, para almacenar el color que tiene
la pieza que está en una casilla, siendo que podría no haber ninguna
pieza en ella.

La verdad es que en C los tipos enumerados son una farsa. Al declarar
una variable de tipo \lstinline!enum color!, realmente estamos
declarando una variable entera, y los valores \lstinline!BLANCO!,
\lstinline!NEGRO! y \lstinline!VACIO! son en realidad los enteros 0, 1,
y 2.

Al compilador le da lo mismo si uno mezcla los valores enumerados con
los enteros, y no descubrirá ningún error que podamos haber cometido. Al
final, usar un tipo enumerado sirve sólo para hacer que el código sea
más fácil de comprender. Pero si cometemos alguna barbaridad como
\lstinline!color = -9000!, que probablemente es un error lógico de
nuestro programa, el compilador hará oídos sordos.

En nuestro programa, nosotros nos aprovechamos de la dualidad
enum-entero para cambiar el turno después de cada jugada. Lo lógico
habría sido hacerlo de este modo:

\begin{lstlisting}
if (turno == BLANCO)
    turno = NEGRO;
else if (turno == NEGRO)
    turno = BLANCO;
\end{lstlisting}

Pero nosotros sabemos que \lstinline!BLANCO! y \lstinline!NEGRO! son 0 y
1, por lo que podemos abreviarlo ingeniosamente (pero no más
claramente):

\begin{lstlisting}
turno = 1 - turno;
\end{lstlisting}

\section{Arreglos bidimensionales}

No debería ser ningún misterio que un \textbf{arreglo bidimensional} es
un arreglo cuyos elementos están numerados por dos índices en lugar de
uno.

Es bastante evidente que, dada la forma que escogimos para representar
las piezas, la mejor manera de representar un tablero de ajedrez en
nuestro programa es usar un arreglo bidimensional de 8 × 8 caracteres:

\begin{lstlisting}
char tablero[8][8];
\end{lstlisting}

Esto hace que tengamos 64 variables de tipo \lstinline!char! a nuestra
disposición, indexadas desde \lstinline!tablero[0][0]! hasta
\lstinline!tablero[7][7]!.

La manera de indexar correctamente un elemento del tablero es usar la
sintaxis \lstinline!tablero[fila][columna]!. Es incorrecto usar la
sintaxis \lstinline!tablero[fila, columna]! que se usa en otros
lenguajes de programación.

Por supuesto, se pueden crear arreglos de todas las dimensiones que uno
quiera, que no necesariamente deben ser de los mismos tamaños:

\begin{lstlisting}
int milimetros_lluvia[100][12][31][24];
                  /* anno mes dia hora */
\end{lstlisting}

\section{Variables globales}

Las variables \lstinline!tablero! y \lstinline!turno! no fueron
declaradas dentro de ninguna función, sino al principio del programa.
Ambas son, pues, \textbf{variables globales}, y por lo mismo pueden ser
usadas desde cualquier parte del programa.

Las variables globales existen desde que el programa comienza hasta que
termina. Jamás son destruidas ni borradas durante la ejecución.

En contraste, las variables declaradas dentro de una función son
\textbf{variables locales}: sólo pueden ser usadas dentro de la función,
son creadas al llamar la función y destruidas cuando la función retorna.

Nuestro programa es más bien pequeño, y por lo tanto las variables
globales no entorpecen el entendimiento. Al contrario: sabemos que sólo
hay un juego en curso (que tiene un tablero y un jugador de turno), por
lo que tener que pasar explícitamente el tablero y el turno a cada
función haría que el código fuera más engorroso. En este caso es
apropiado usar variables globales.

Sin embargo, al desarrollar aplicaciones grandes, uno debe evitar usar
las variables globales como medio de comunicación entre partes del
programa. Idealmente, cada función debería recibir toda la información
que necesita a través de sus parámetros, y entregar sus resultados como
valor de retorno.

Al usar información global, el comportamiento de un trozo de código
puede ser diferente dependiendo del estado de variables que son
asignadas en partes bien alejadas del código. Esto hace que los
programas sean más difícil de entender (porque hay que figurarse en la
cabeza cuál es el estado global), y los errores más difíciles de
depurar.

Si usamos sólo información local (variables locales, parámetros, valores
de retorno) entonces todo el comportamiento de una sección de programa
está determinado por información que se encuentra cercana a ella en el
código.

\section{Funciones que no retornan nada}

La sintaxis para crear una función en C exige indicar de qué tipo es su
valor de retorno. Sin embargo, es posible crear una función que no
retorne nada.

Para esto, hay que poner que su tipo de retorno es \lstinline!void!, que
se puede interpretar como «ningún tipo».

Hay varias razones por la que una función podría no retornar nada:

\begin{itemize}
\item
  la función hace entrada o salida (como \lstinline!imprimir_tablero!),
\item
  la función actúa sobre variables globales (como
  \lstinline!mover_pieza!),
\item
  la función debe «retornar» más de un valor, por lo que se usan los
  parámetros para entregar los valores (como \lstinline!leer_jugada!,
  explicado más adelante).
\end{itemize}

\section{Macros con parámetros}

Ya vimos que una macro es una sustitución textual que se hace en el
código previo a la compilación. Las macros además pueden recibir
parámetros, lo que las convierte en una especie de plantilla para hacer
sustituciones en el programa.

Como en nuestro programa del ajedrez tenemos que hacer varios ciclos que
vayan de 0 a 7 para recorrer el tablero, decidimos crear una macro
llamada \lstinline!FOR! que sea equivalente a hacer un ciclo
\lstinline!for! de 0 a 7 con alguna variable:

\begin{lstlisting}
#define FOR(var)  for (var = 0; var < 8; ++var)
\end{lstlisting}

Como parámetro, debemos pasarle a \lstinline!FOR! el nombre de la
variable que queremos usar como contador en nuestro ciclo. Note que no
estamos pasando el valor de la variable: la sustitución es meramente
textual.

En esencia estamos modificando la sintaxis del lenguaje. En general
\textbf{es una mala práctica hacer cosas como ésta}. Como la sustitución
es puramente textual y nunca se verifica que tenga sentido, esto puede
causar errores muy extraños si no se programa con cuidado. También
estamos haciendo más difícil a otros programadores entender nuestro
código, ya que ellos están familiarizados con la sintaxis del lenguaje
pero no con nuestras construcciones.

Uno puede ponerse muy creativo para crear macros. Casi nunca es buena
idea ceder a la tentación. Sólo hay que hacerlo cuando en efecto se
logra hacer que el programa resulte más legible. ¿Cree usted que lo
conseguimos con este ejemplo?

\section{Declaraciones de funciones}

El compilador analiza el código del programa de arriba hacia abajo. Es
necesario que todos los nombres (como variables, tipos y funciones) ya
hayan sido declarados antes de que aparezcan en el código.

Por esto mismo, cuando creábamos funciones, lo hacíamos antes de la
función \lstinline!main!. De otro modo, el compilador no sabría que las
funciones que llamamos desde \lstinline!main! existen.

En el caso de las funciones, hay que distinguir entre dos cosas:

\begin{itemize}
\item
  la \textbf{declaración} de la función, que consiste en especificar su
  nombre, su tipo de retorno y los tipos de los parámetros para que el
  compilador los conozca, y
\item
  la \textbf{definición} de la función, que es especificar cuál es el
  código de la función.
\end{itemize}

En el programa del ajedrez, las funciones están declaradas pero no
definidas al principio del código. Gracias a esto, las podemos llamar
desde \lstinline!main!. El compilador sabrá exactamente a qué nos
referimos cuando decimos \lstinline!color_pieza! o
\lstinline!juego_terminado!, y podrá verificar que estos nombres están
siendo usados correctamente.

Aún así, las funciones deben estar definidas en alguna otra parte
(única) del programa para que la compilación pueda ser completada.

Para declarar las funciones, no es necesario explicitar los nombres de
los parámetros, pero está permitido hacerlo. Las siguientes dos
declaraciones son válidas:

\begin{lstlisting}
float potencia(float, int);
float potencia(float base, int exponente);
\end{lstlisting}

En C, se le llama \textbf{prototipo} a la declaración de una función.
Los tipos de retorno y de los parámetros conforman la \textbf{firma} de
la función.

Las cabeceras \lstinline!.h! que incluímos siempre en nuestros programas
contienen sólo las declaraciones de las funciones que proveen, no las
definiciones. Al compilar nuestro código, al compilador no le importa
qué hacen esas funciones, sino solamente cuáles son sus firmas. Recién
en la fase final de la compilación (llamada
\href{http://en.wikipedia.org/wiki/Linker\_(computing)}{enlazado}) el
compilador se encarga de averiguar dónde está el código (ya compilado)
de esas funciones.

Si una función está definida pero no declarada, entonces la definición
actúa también como declaración. Si una función está declarada antes de
ser definida, entonces las firmas del prototipo y de la definición deben
coincidir.

\section{Paso de parámetros por referencia}

En C, cuando una función es llamada, sus parámetros reciben una copia de
los valores que fueron pasados como argumentos. Todos los cambios que
uno haga a los parámetros no se verán reflejados afuera de la función:

Se dice que C hace paso de parámetros \textbf{por valor}, en oposición a
lenguajes donde el paso de parámetros es \textbf{por referencia}.

En C se puede emular el paso por referencia para que sí se pueda
modificar una variable definida fuera de la función. Para que esto
funcione, no hay que pasar a la función el valor de la variable, sino su
dirección de memoria. Por supuesto, el parámetro debe ser ahora un
puntero (que es el tipo apropiado para guardar direcciones de memoria):

Compare ambos programas y asegúrese de entender las diferencias.

Note que la función debe derreferenciar el parámetro cada vez que haya
que referirse a la variable original.

Una razón común para usar paso por referencia es permitir que una
función entregue más de un resultado. Por ejemplo, nuestra función
\lstinline!leer_jugada! le pide al usuario ingresar cuatro datos que
debe usar el programa. Como en C no se pueden retornar 4 valores (a no
ser que se los junte en una estructura), es mejor pasarle las variables
por referencia. Esto es como decirle a la función «deja aquí los
resultados».

Otra razón para usar paso por referencia es evitar que se copien muchos
datos cuando los parámetros son estructuras o arreglos grandes, incluso
si no es necesario modificarlos dentro de la función:

\begin{lstlisting}
struct grande {
    int a, b, c;    /* 4 bytes cada uno */
    float x, y;     /* 4 bytes cada uno */
    char z[100];    /* 100 bytes */
} g;

fv(g);   /* se copian 120 bytes */
fr(&g);  /* se copian 4 bytes (taman~o de un puntero) */
\end{lstlisting}

Se puede indicar al compilador que la función no modificará la variable
original usando el calificador \lstinline!const!:

\begin{lstlisting}
void fr(const struct grande *g) {
    printf("%s\n", (*g).z);
}
\end{lstlisting}

\section{Conversiones de caracteres}

Los valores de tipo \lstinline!char! en realidad son enteros. El mapeo
entre símbolos y enteros está dado por el código ASCII\_.

Al igual como hicimos con los tipos enumerados, podemos mezclar
libremente enteros y caracteres al hacer operaciones. Esto no es así en
otros lenguajes con sistemas de tipos más
\href{http://en.wikipedia.org/wiki/Strong\_typing}{fuertes}. Recordemos
que en Python era un error sumar un entero a un caracter:

\begin{lstlisting}
>>> 'a' + 3
Traceback (most recent call last):
  File "<console>", line 1, in <module>
TypeError: cannot concatenate 'str' and 'int' objects
\end{lstlisting}

Mientras que en C sí está permitido:

\begin{lstlisting}
printf("%c", 'a' + 3);
/* esto imprime el caracter d */
\end{lstlisting}

En este ejemplo, estamos sumando 3 al código ASCII de \lstinline!'a'!, y
pasa que el resultado es el código ASCII de \lstinline!'d'!. Estamos
usando la propiedad del código ASCII de que las letras minúsculas en
orden alfabético tienen códigos correlativos. Lo mismo ocurre con las
mayúsculas y con los dígitos del \lstinline!'0'! al \lstinline!'9'!.

Sin necesidad de recordar cuál es el código exacto de cada símbolo, es
posible usar estas propiedades para hacer conversiones entre caracteres
y enteros. Por ejemplo, para llevar un caracter entre \lstinline!'0'! y
\lstinline!'9'! a su entero correspondiente, basta con restarle el valor
\lstinline!'0'!:

\begin{lstlisting}
'0' - '0' == 0
'1' - '0' == 1
'2' - '0' == 2
etc.
\end{lstlisting}

De manera análoga, podemos obtener la posición de una letra en el
abecedario restándole el código de la letra \emph{a}:

\begin{lstlisting}
'a' - 'a' == 0
'b' - 'a' == 1
'c' - 'a' == 2
etc.
\end{lstlisting}

Nosotros aprovechamos estas propiedades en la función
\lstinline!leer_jugada!. El usuario ingresa las coordenadas de una
casilla como un par de caracteres letra-dígito, mientras que el programa
representa las casillas como pares entero-entero. Para convertir a
entero, simplemente restamos \lstinline!'a'! o \lstinline!'0'! según
corresponda.

Otro truco muy usado para convertir un caracter de minúsculas a
mayúsculas es el siguiente:

\begin{lstlisting}
char may = min - 'a' + 'A';
\end{lstlisting}

Este truco tiene su encanto, pero no es muy confiable (¿qué pasa si
\lstinline!min! ya está en mayúsculas?). Siempre es mejor usar la
función \lstinline!toupper! (provista por \lstinline!ctype.h!), como
hicimos en \lstinline!inicializar_tablero!.

\lstinline!ctype.h! provee varias otras funciones para operar sobre
caracteres.
