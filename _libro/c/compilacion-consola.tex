\chapter{Compilación de programas en la consola}

Los sistemas operativos tipo Unix, como Linux o Mac OS, vienen de
fábrica con entornos de consola. En Mac la consola del sistema se llama
Terminal. En Linux tiene varios para escoger: Gnome Terminal, Konsole o
XTerm, entre otros.

En Windows usted puede instalar un entorno tipo Unix (por ejemplo
Cygwin) para disponer de una consola funcional para probar los ejemplos
de este apunte.

En los ejemplos de sesiones de consola que se muestran a continuación,
el signo \lstinline!$! representa lo que sea que aparezca en su terminal
para indicarle que ya puede ingresar una instrucción. El signo
\lstinline!$! no es parte del comando, y no debe ser escrito en el
teclado.

Las líneas que no comienzan con \lstinline!$! muestran la salida del
último comando ejecutado.

Algunos comandos útiles para manejarse en la consola:

\begin{itemize}
\item
  \lstinline!mkdir! crea un directorio nuevo,
\item
  \lstinline!cd! sirve para cambiar el directorio actual,
\item
  \lstinline!pwd! muestra el directorio actual,
\item
  \lstinline!ls! muestra los archivos en el directorio actual.
\end{itemize}

\begin{lstlisting}
$ mkdir programacion
$ ls
programacion
$ cd programacion
$ pwd
/home/usuario/programacion
$
\end{lstlisting}

\section{Compilación usando GCC}

\href{http://gcc.gnu.org/}{GCC} es el más popular de los compiladores de
C. GCC es libre y multiplataforma, lo que asegura que puede instalarlo y
usarlo casi en cualquier sistema, y sin necesidad de pagar una licencia.

Para compilar un programa llamado \lstinline!test.c! que está en el
mismo directorio donde estamos parados, hay que hacerlo así:

\begin{lstlisting}
$ gcc test.c -o test
\end{lstlisting}

Esta instrucción indica a GCC que debe compilar el archivo fuente
\lstinline!test.c!, y crear un binario ejecutable llamado
\lstinline!test!.

La opción \lstinline!-o! señala que lo que viene a continuación es el
nombre del archivo de salida (\emph{output}) que debe ser creado por el
compilador. Si usted omite esta opción, entonces el ejecutable será
creado con el nombre \lstinline!a.out!:

\begin{lstlisting}
$ ls
test.c
$ gcc test.c
$ ls
a.out  test.c
\end{lstlisting}

No sea flojo: siempre especifique el nombre del archivo de salida.

\section{Compilación usando make}

Make es un programa muy útil que sí nos permite ser flojos. Make conoce
las maneras más comunes de compilar programas en C (y en otros lenguajes
también), y hace el trabajo por nosotros.

Para programas pequeños como los que haremos aquí no es mucho lo que
ayuda, pero conviene aprender a usarlo porque es muy utilizado para
automatizar la compilación de proyectos más grandes.

Al ejecutar make, no se debe indicar cómo compilar, sino lo que uno
quiere obtener. En nuestro caso, es el ejecutable \lstinline!test!:

\begin{lstlisting}
$ ls
test.c
$ make test
cc     test.c   -o test
$ ls
test  test.c
$
\end{lstlisting}

Make sabe que si queremos obtener un binario llamado \lstinline!test!, y
en el directorio hay un archivo fuente llamado \lstinline!test.c!, lo
que hay que hacer es invocar al compilador de C.

Por omisión, make usa el compilador \lstinline!cc! cuando le toca
compilar un archivo \lstinline!.c!. De todos modos, lo más probable es
que en su sistema \lstinline!cc! sea un enlace simbólico a
\lstinline!gcc!:

\begin{lstlisting}
$ which cc
/usr/bin/cc
$ ls -o /usr/bin/cc
lrwxrwxrwx. 1 root 3 ene  5 22:01 /usr/bin/cc -> gcc
$
\end{lstlisting}

Para indicar explícitamente a make que utilice el compilador GCC (o
cualquier otro) para compilar, se debe asignar el nombre del compilador
a la variable de entorno \lstinline!CC! usando la instrucción
\lstinline!export!:

\begin{lstlisting}
$ make test
cc     test.c   -o test
$ rm test
$ export CC=gcc
$ make test
gcc     test.c   -o test
$
\end{lstlisting}

Una de las gracias de make es que sólo hace la compilación si es que el
archivo con el código ha sido modificado desde la última vez que se
compiló. Si no ha habido cambios desde entonces, make no hace nada:

\begin{lstlisting}
$ ls
test.c
$ make test
cc     test.c   -o test
$ make test
make: `test' esta actualizado.
$
\end{lstlisting}

\section{Ejecución de un programa}

Para ejecutar un programa, se debe escribir su nombre precedido de
\lstinline!./! desde el mismo directorio donde quedó el ejecutable:

\begin{lstlisting}
$ ./test
Felicidades! Usted ha ejecutado el programa test.
\end{lstlisting}

Ahora que sabemos compilar y ejecutar programas, analizaremos varios
programas en orden creciente de complejidad, e iremos presentando
gradualmente las características de C.
