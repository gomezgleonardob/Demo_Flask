\chapter{Calcular la edad del usuario}

El siguiente programa le pide al usuario ingresar su año de nacimiento y
el año actual. A continuación, le muestra cuál es su edad:

Como siempre, el código del programa debe estar incluído dentro de una
función llamada \lstinline!main!, y la última sentencia del programa
debe ser \lstinline!return 0!.

Escriba, compile y ejecute este programa.

\section{Declaración de variables}

Este programa utiliza tres variables, llamadas \lstinline!nacimiento!,
\lstinline!actual! y \lstinline!edad!.

En Python, las variables eran creadas automáticamente al momento de
asignarlas por primera vez:

\begin{lstlisting}
nacimiento = int(raw_input("Ingrese su anno de nacimiento: "))
actual = int(raw_input("Ingrese el anno actual: "))
edad = actual - nacimiento
\end{lstlisting}

En C no es así. Las variables deben ser \textbf{declaradas} antes de ser
usadas. Además, uno debe indicar de qué tipo serán los datos que se
almacenarán en cada variable. Una variable sólo puede almacenar valores
de un único tipo.

Las tres primeras sentencias del programa declaran las variables
\lstinline!nacimiento!, \lstinline!actual! y \lstinline!edad! para
almacenar valores de tipo \lstinline!int! (entero).

En C, todas las declaraciones deben cumplir con esta sintaxis:

\begin{lstlisting}
tipo variable;
\end{lstlisting}

\subsubsection{¿Por qué es necesario declarar las variables?}

Una característica del lenguaje C es que entrega al programador el poder
(y la responsabilidad) de decidir muy de cerca cómo usar la memoria del
computador. En Python, al contrario, el intérprete decide por uno
cuándo, cómo y cuánta memoria el programa utilizará, lo que es muy
conveniente a la hora de programar pero que puede conducir a un uso
ineficiente de los recursos disponibles en ciertas ocasiones.

En nuestro programa de ejemplo, el compilador analizará el código y
sabrá que el programa sólo almacenará tres valores, y que cada uno sólo
necesitará el espacio suficiente para guardar un número entero. ¡Todo
esto ocurre antes de que el programa sea siquiera ejecutado por primera
vez!

\section{Entrada con formato usando scanf}

Ya conocimos la función \lstinline!printf!, que sirve para imprimir un
(único) string por pantalla.

Para recibir la entrada del programa se utiliza la función
\textbf{scanf}, cuyo uso puede parecer un poco extraño al principio.

El primer parámetro de la función \lstinline!scanf! es un string que
describe cuál es el formato en el que estará representado el valor a
ingresar. En este ejemplo, el string \lstinline!"%d"! indica que el
valor que será leído debe ser interpretado como un número entero en
representación decimal (dígitos del 0 al 9, posiblemente con un signo al
principio). Por supuesto, hay muchos otros descriptores de formato.

El segundo parámetro debe indicar \textbf{en qué lugar de la memoria del
computador se debe guardar el valor ingresado}. Note que aquí no se pone
la variable a secas, sino que antecedida de un signo \lstinline!&!. La
distinción es importante:

\begin{itemize}
\item
  \lstinline!nacimiento! es el valor que tiene la variable
  \lstinline!nacimiento!,
\item
  \lstinline!&nacimiento! es la ubicación en la memoria de la variable
  \lstinline!nacimiento!.
\end{itemize}

El operador \lstinline!&! se lee como «la dirección de». Más adelante
veremos qué significa esto.

En resumen, la sentencia:

\begin{lstlisting}
scanf("%d", &nacimiento);
\end{lstlisting}

es equivalente a la siguiente sentencia en Python:

\begin{lstlisting}
nacimiento = int(raw_input())
\end{lstlisting}

\section{Salida con formato usando printf}

La función \lstinline!printf! imprime sólo strings, no enteros. Sin
embargo, es posible insertar enteros dentro del mensaje usando
descriptores de formato idénticos a los de la función \lstinline!scanf!.

En las posiciones del string en las que se desea mostrar un número
entero, debe insertarse el texto \lstinline!%d!. Luego, cada uno de los
valores enteros por imprimir deben ser pasados como parámetros
adicionales a la función.

Los siguientes ejemplos muestran usos correctos e incorrectos de
\lstinline!printf!. Haga el ejercicio de darse cuenta de los errores:

\begin{lstlisting}
/* Correctos */
printf("Hola mundo\n");
printf("Usted tiene %d annos.", edad);
printf("Usted tiene %d annos.\n", edad);
printf("Usted tiene 18 annos.");
printf("Usted tiene %d annos.", 18);
printf("Usted tiene %d annos y %d meses.", edad, meses);

/* Incorrectos */
printf("Usted tiene %d annos.");
printf("Usted tiene annos.", edad);
printf("Usted tiene annos.", 18);
printf("Usted tiene", edad, "annos.");
printf("Usted tiene edad annos.");
printf("Usted tiene"); printf(edad); printf("annos.");
printf("Usted tiene %d annos y %d meses.", edad);
\end{lstlisting}

\section{Ejercicio}

Escriba un programa que pregunte al usuario las notas de sus cuatro
certámenes, y le muestre cuál es su promedio, con decimales:

Para declarar una variable de tipo real, se debe indicar que el tipo es
\lstinline!float!.

Para leer y para mostrar un número real con decimales, se usa el
descriptor de formato \lstinline!%f!.
