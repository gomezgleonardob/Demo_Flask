\chapter{Hola mundo}

El siguiente es un programa que le muestra al usuario el mensaje «Hola
mundo»:

Escriba este programa en su editor favorito. ¡No copie y pegue,
escríbalo a mano! Así se irá familiarizando con la sintaxis del
lenguaje. Guarde el programa con el nombre \lstinline!hola.c!.

Compile el programa. Si el programa no compila, entonces cometió algún
error al transcribirlo. Lea el mensaje de error del compilador, descubra
los errores, y arregle el programa todas las veces necesarias hasta que
compile y se ejecute correctamente.

\section{Función main}

En un programa en C, todas las instrucciones deben estar dentro de una
función.

Todos los programas deben tener una función con nombre \lstinline!main!.
El código que está dentro de la función \lstinline!main! es lo que hace
el programa cuando es ejecutado.

La línea \lstinline!int main()! es la que indica que el código que viene
a continuación, entre los paréntesis de llave (\lstinline!{! y
\lstinline!}!) es parte de esta función.

Cuando la función \lstinline!main! retorna un valor, entonces el
programa se termina. El valor que debe retornar debe ser un entero (esto
es lo que significa el \lstinline!int! de la definición). Si el programa
se ejecuta correctamente, entonces debe retornarse cero. Si se retorna
un valor distinto de cero, se está indicando que ocurrió algún error
durante la ejecución del programa.

Como regla general, al final de la función \lstinline!main! siempre debe
ir un \lstinline!return 0!, como en el ejemplo.

En C, todas las sentencias deben obligatoriamente terminar con un punto
y coma.

\section{Salida usando printf}

La función \lstinline!printf! muestra un mensaje en la pantalla. El
mensaje debe ser un string. Los strings literales se representan entre
comillas dobles (¡nunca entre comillas simples!):

\begin{lstlisting}
"Hola mundo\n"
\end{lstlisting}

A diferencia del \lstinline!print! de Python, \lstinline!printf! no pone
un salto de línea al final del mensaje. El salto de línea debe ser
agregado explícitamente usando su representación \lstinline!\n!. Por
ejemplo, el siguiente código también imprime el mensaje «Hola mundo» en
una única línea, y pone un salto de línea al final:

\begin{lstlisting}
printf("Ho");
printf("la mu");
printf("ndo\n");
\end{lstlisting}

\section{Inclusión de cabeceras}

Técnicamente, la función \lstinline!printf! no es parte del lenguaje
(como lo es el \lstinline!print! de Python), sino que es parte de la
\href{http://es.wikipedia.org/wiki/Biblioteca\_est\%C3\%A1ndar\_de\_C}{biblioteca
estándar} de C.

La biblioteca estándar es una colección de funciones, constantes y tipos
que son comúnmente usados en la mayoría de los programas. Basta con
tener instalado el compilador de C para tener toda la biblioteca
estándar a disposición.

Para poder usar una función en un programa, ella debe ser declarada en
alguna parte del código. Afortunadamente, la biblioteca estándar provee
\href{http://es.wikipedia.org/wiki/Archivo\_de\_cabecera}{archivos de
cabecera} (\emph{header files}) que contienen las declaraciones de todas
sus funciones, organizadas de acuerdo a su utilidad. Los archivos de
cabecera suelen tener nombres terminados en la extensión \lstinline!.h!.

La función \lstinline!printf! está declarada en el archivo de cabecera
\lstinline!stdio.h!, que agrupa las funciones de entrada y salida
(«\lstinline!io!») de la biblioteca estándar («\lstinline!std!»). Para
poder usar la función, hay que incluir la cabecera usando la directiva
\lstinline!#include!, tal como se muestra en el ejemplo.

Más adelante veremos otros archivos de cabecera. También podremos crear
nuestras propias bibliotecas, que requerirán sus respectivas cabeceras.

\section{Ejercicios}

Modifique el programa para que imprima el siguiente haiku:

\begin{lstlisting}
Al programar,
cuando digo "hola mundo",
aprendo C.
\end{lstlisting}

Puede hacerlo con un único \lstinline!printf! o con varios. Averigüe
cómo hacer para imprimir las comillas.

¿Qué ocurre si la función tiene un nombre diferente de \lstinline!main!?
¿Qué ocurre si omite la línea del \lstinline!include!? ¿Qué ocurre si no
pone el \lstinline!return 0!? Haga la prueba.
